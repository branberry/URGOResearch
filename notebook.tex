
% Default to the notebook output style

    


% Inherit from the specified cell style.




    
\documentclass[11pt]{article}

    
    
    \usepackage[T1]{fontenc}
    % Nicer default font (+ math font) than Computer Modern for most use cases
    \usepackage{mathpazo}

    % Basic figure setup, for now with no caption control since it's done
    % automatically by Pandoc (which extracts ![](path) syntax from Markdown).
    \usepackage{graphicx}
    % We will generate all images so they have a width \maxwidth. This means
    % that they will get their normal width if they fit onto the page, but
    % are scaled down if they would overflow the margins.
    \makeatletter
    \def\maxwidth{\ifdim\Gin@nat@width>\linewidth\linewidth
    \else\Gin@nat@width\fi}
    \makeatother
    \let\Oldincludegraphics\includegraphics
    % Set max figure width to be 80% of text width, for now hardcoded.
    \renewcommand{\includegraphics}[1]{\Oldincludegraphics[width=.8\maxwidth]{#1}}
    % Ensure that by default, figures have no caption (until we provide a
    % proper Figure object with a Caption API and a way to capture that
    % in the conversion process - todo).
    \usepackage{caption}
    \DeclareCaptionLabelFormat{nolabel}{}
    \captionsetup{labelformat=nolabel}

    \usepackage{adjustbox} % Used to constrain images to a maximum size 
    \usepackage{xcolor} % Allow colors to be defined
    \usepackage{enumerate} % Needed for markdown enumerations to work
    \usepackage{geometry} % Used to adjust the document margins
    \usepackage{amsmath} % Equations
    \usepackage{amssymb} % Equations
    \usepackage{textcomp} % defines textquotesingle
    % Hack from http://tex.stackexchange.com/a/47451/13684:
    \AtBeginDocument{%
        \def\PYZsq{\textquotesingle}% Upright quotes in Pygmentized code
    }
    \usepackage{upquote} % Upright quotes for verbatim code
    \usepackage{eurosym} % defines \euro
    \usepackage[mathletters]{ucs} % Extended unicode (utf-8) support
    \usepackage[utf8x]{inputenc} % Allow utf-8 characters in the tex document
    \usepackage{fancyvrb} % verbatim replacement that allows latex
    \usepackage{grffile} % extends the file name processing of package graphics 
                         % to support a larger range 
    % The hyperref package gives us a pdf with properly built
    % internal navigation ('pdf bookmarks' for the table of contents,
    % internal cross-reference links, web links for URLs, etc.)
    \usepackage{hyperref}
    \usepackage{longtable} % longtable support required by pandoc >1.10
    \usepackage{booktabs}  % table support for pandoc > 1.12.2
    \usepackage[inline]{enumitem} % IRkernel/repr support (it uses the enumerate* environment)
    \usepackage[normalem]{ulem} % ulem is needed to support strikethroughs (\sout)
                                % normalem makes italics be italics, not underlines
    

    
    
    % Colors for the hyperref package
    \definecolor{urlcolor}{rgb}{0,.145,.698}
    \definecolor{linkcolor}{rgb}{.71,0.21,0.01}
    \definecolor{citecolor}{rgb}{.12,.54,.11}

    % ANSI colors
    \definecolor{ansi-black}{HTML}{3E424D}
    \definecolor{ansi-black-intense}{HTML}{282C36}
    \definecolor{ansi-red}{HTML}{E75C58}
    \definecolor{ansi-red-intense}{HTML}{B22B31}
    \definecolor{ansi-green}{HTML}{00A250}
    \definecolor{ansi-green-intense}{HTML}{007427}
    \definecolor{ansi-yellow}{HTML}{DDB62B}
    \definecolor{ansi-yellow-intense}{HTML}{B27D12}
    \definecolor{ansi-blue}{HTML}{208FFB}
    \definecolor{ansi-blue-intense}{HTML}{0065CA}
    \definecolor{ansi-magenta}{HTML}{D160C4}
    \definecolor{ansi-magenta-intense}{HTML}{A03196}
    \definecolor{ansi-cyan}{HTML}{60C6C8}
    \definecolor{ansi-cyan-intense}{HTML}{258F8F}
    \definecolor{ansi-white}{HTML}{C5C1B4}
    \definecolor{ansi-white-intense}{HTML}{A1A6B2}

    % commands and environments needed by pandoc snippets
    % extracted from the output of `pandoc -s`
    \providecommand{\tightlist}{%
      \setlength{\itemsep}{0pt}\setlength{\parskip}{0pt}}
    \DefineVerbatimEnvironment{Highlighting}{Verbatim}{commandchars=\\\{\}}
    % Add ',fontsize=\small' for more characters per line
    \newenvironment{Shaded}{}{}
    \newcommand{\KeywordTok}[1]{\textcolor[rgb]{0.00,0.44,0.13}{\textbf{{#1}}}}
    \newcommand{\DataTypeTok}[1]{\textcolor[rgb]{0.56,0.13,0.00}{{#1}}}
    \newcommand{\DecValTok}[1]{\textcolor[rgb]{0.25,0.63,0.44}{{#1}}}
    \newcommand{\BaseNTok}[1]{\textcolor[rgb]{0.25,0.63,0.44}{{#1}}}
    \newcommand{\FloatTok}[1]{\textcolor[rgb]{0.25,0.63,0.44}{{#1}}}
    \newcommand{\CharTok}[1]{\textcolor[rgb]{0.25,0.44,0.63}{{#1}}}
    \newcommand{\StringTok}[1]{\textcolor[rgb]{0.25,0.44,0.63}{{#1}}}
    \newcommand{\CommentTok}[1]{\textcolor[rgb]{0.38,0.63,0.69}{\textit{{#1}}}}
    \newcommand{\OtherTok}[1]{\textcolor[rgb]{0.00,0.44,0.13}{{#1}}}
    \newcommand{\AlertTok}[1]{\textcolor[rgb]{1.00,0.00,0.00}{\textbf{{#1}}}}
    \newcommand{\FunctionTok}[1]{\textcolor[rgb]{0.02,0.16,0.49}{{#1}}}
    \newcommand{\RegionMarkerTok}[1]{{#1}}
    \newcommand{\ErrorTok}[1]{\textcolor[rgb]{1.00,0.00,0.00}{\textbf{{#1}}}}
    \newcommand{\NormalTok}[1]{{#1}}
    
    % Additional commands for more recent versions of Pandoc
    \newcommand{\ConstantTok}[1]{\textcolor[rgb]{0.53,0.00,0.00}{{#1}}}
    \newcommand{\SpecialCharTok}[1]{\textcolor[rgb]{0.25,0.44,0.63}{{#1}}}
    \newcommand{\VerbatimStringTok}[1]{\textcolor[rgb]{0.25,0.44,0.63}{{#1}}}
    \newcommand{\SpecialStringTok}[1]{\textcolor[rgb]{0.73,0.40,0.53}{{#1}}}
    \newcommand{\ImportTok}[1]{{#1}}
    \newcommand{\DocumentationTok}[1]{\textcolor[rgb]{0.73,0.13,0.13}{\textit{{#1}}}}
    \newcommand{\AnnotationTok}[1]{\textcolor[rgb]{0.38,0.63,0.69}{\textbf{\textit{{#1}}}}}
    \newcommand{\CommentVarTok}[1]{\textcolor[rgb]{0.38,0.63,0.69}{\textbf{\textit{{#1}}}}}
    \newcommand{\VariableTok}[1]{\textcolor[rgb]{0.10,0.09,0.49}{{#1}}}
    \newcommand{\ControlFlowTok}[1]{\textcolor[rgb]{0.00,0.44,0.13}{\textbf{{#1}}}}
    \newcommand{\OperatorTok}[1]{\textcolor[rgb]{0.40,0.40,0.40}{{#1}}}
    \newcommand{\BuiltInTok}[1]{{#1}}
    \newcommand{\ExtensionTok}[1]{{#1}}
    \newcommand{\PreprocessorTok}[1]{\textcolor[rgb]{0.74,0.48,0.00}{{#1}}}
    \newcommand{\AttributeTok}[1]{\textcolor[rgb]{0.49,0.56,0.16}{{#1}}}
    \newcommand{\InformationTok}[1]{\textcolor[rgb]{0.38,0.63,0.69}{\textbf{\textit{{#1}}}}}
    \newcommand{\WarningTok}[1]{\textcolor[rgb]{0.38,0.63,0.69}{\textbf{\textit{{#1}}}}}
    
    
    % Define a nice break command that doesn't care if a line doesn't already
    % exist.
    \def\br{\hspace*{\fill} \\* }
    % Math Jax compatability definitions
    \def\gt{>}
    \def\lt{<}
    % Document parameters
    \title{Numerical Quadrature}
    
    
    

    % Pygments definitions
    
\makeatletter
\def\PY@reset{\let\PY@it=\relax \let\PY@bf=\relax%
    \let\PY@ul=\relax \let\PY@tc=\relax%
    \let\PY@bc=\relax \let\PY@ff=\relax}
\def\PY@tok#1{\csname PY@tok@#1\endcsname}
\def\PY@toks#1+{\ifx\relax#1\empty\else%
    \PY@tok{#1}\expandafter\PY@toks\fi}
\def\PY@do#1{\PY@bc{\PY@tc{\PY@ul{%
    \PY@it{\PY@bf{\PY@ff{#1}}}}}}}
\def\PY#1#2{\PY@reset\PY@toks#1+\relax+\PY@do{#2}}

\expandafter\def\csname PY@tok@w\endcsname{\def\PY@tc##1{\textcolor[rgb]{0.73,0.73,0.73}{##1}}}
\expandafter\def\csname PY@tok@c\endcsname{\let\PY@it=\textit\def\PY@tc##1{\textcolor[rgb]{0.25,0.50,0.50}{##1}}}
\expandafter\def\csname PY@tok@cp\endcsname{\def\PY@tc##1{\textcolor[rgb]{0.74,0.48,0.00}{##1}}}
\expandafter\def\csname PY@tok@k\endcsname{\let\PY@bf=\textbf\def\PY@tc##1{\textcolor[rgb]{0.00,0.50,0.00}{##1}}}
\expandafter\def\csname PY@tok@kp\endcsname{\def\PY@tc##1{\textcolor[rgb]{0.00,0.50,0.00}{##1}}}
\expandafter\def\csname PY@tok@kt\endcsname{\def\PY@tc##1{\textcolor[rgb]{0.69,0.00,0.25}{##1}}}
\expandafter\def\csname PY@tok@o\endcsname{\def\PY@tc##1{\textcolor[rgb]{0.40,0.40,0.40}{##1}}}
\expandafter\def\csname PY@tok@ow\endcsname{\let\PY@bf=\textbf\def\PY@tc##1{\textcolor[rgb]{0.67,0.13,1.00}{##1}}}
\expandafter\def\csname PY@tok@nb\endcsname{\def\PY@tc##1{\textcolor[rgb]{0.00,0.50,0.00}{##1}}}
\expandafter\def\csname PY@tok@nf\endcsname{\def\PY@tc##1{\textcolor[rgb]{0.00,0.00,1.00}{##1}}}
\expandafter\def\csname PY@tok@nc\endcsname{\let\PY@bf=\textbf\def\PY@tc##1{\textcolor[rgb]{0.00,0.00,1.00}{##1}}}
\expandafter\def\csname PY@tok@nn\endcsname{\let\PY@bf=\textbf\def\PY@tc##1{\textcolor[rgb]{0.00,0.00,1.00}{##1}}}
\expandafter\def\csname PY@tok@ne\endcsname{\let\PY@bf=\textbf\def\PY@tc##1{\textcolor[rgb]{0.82,0.25,0.23}{##1}}}
\expandafter\def\csname PY@tok@nv\endcsname{\def\PY@tc##1{\textcolor[rgb]{0.10,0.09,0.49}{##1}}}
\expandafter\def\csname PY@tok@no\endcsname{\def\PY@tc##1{\textcolor[rgb]{0.53,0.00,0.00}{##1}}}
\expandafter\def\csname PY@tok@nl\endcsname{\def\PY@tc##1{\textcolor[rgb]{0.63,0.63,0.00}{##1}}}
\expandafter\def\csname PY@tok@ni\endcsname{\let\PY@bf=\textbf\def\PY@tc##1{\textcolor[rgb]{0.60,0.60,0.60}{##1}}}
\expandafter\def\csname PY@tok@na\endcsname{\def\PY@tc##1{\textcolor[rgb]{0.49,0.56,0.16}{##1}}}
\expandafter\def\csname PY@tok@nt\endcsname{\let\PY@bf=\textbf\def\PY@tc##1{\textcolor[rgb]{0.00,0.50,0.00}{##1}}}
\expandafter\def\csname PY@tok@nd\endcsname{\def\PY@tc##1{\textcolor[rgb]{0.67,0.13,1.00}{##1}}}
\expandafter\def\csname PY@tok@s\endcsname{\def\PY@tc##1{\textcolor[rgb]{0.73,0.13,0.13}{##1}}}
\expandafter\def\csname PY@tok@sd\endcsname{\let\PY@it=\textit\def\PY@tc##1{\textcolor[rgb]{0.73,0.13,0.13}{##1}}}
\expandafter\def\csname PY@tok@si\endcsname{\let\PY@bf=\textbf\def\PY@tc##1{\textcolor[rgb]{0.73,0.40,0.53}{##1}}}
\expandafter\def\csname PY@tok@se\endcsname{\let\PY@bf=\textbf\def\PY@tc##1{\textcolor[rgb]{0.73,0.40,0.13}{##1}}}
\expandafter\def\csname PY@tok@sr\endcsname{\def\PY@tc##1{\textcolor[rgb]{0.73,0.40,0.53}{##1}}}
\expandafter\def\csname PY@tok@ss\endcsname{\def\PY@tc##1{\textcolor[rgb]{0.10,0.09,0.49}{##1}}}
\expandafter\def\csname PY@tok@sx\endcsname{\def\PY@tc##1{\textcolor[rgb]{0.00,0.50,0.00}{##1}}}
\expandafter\def\csname PY@tok@m\endcsname{\def\PY@tc##1{\textcolor[rgb]{0.40,0.40,0.40}{##1}}}
\expandafter\def\csname PY@tok@gh\endcsname{\let\PY@bf=\textbf\def\PY@tc##1{\textcolor[rgb]{0.00,0.00,0.50}{##1}}}
\expandafter\def\csname PY@tok@gu\endcsname{\let\PY@bf=\textbf\def\PY@tc##1{\textcolor[rgb]{0.50,0.00,0.50}{##1}}}
\expandafter\def\csname PY@tok@gd\endcsname{\def\PY@tc##1{\textcolor[rgb]{0.63,0.00,0.00}{##1}}}
\expandafter\def\csname PY@tok@gi\endcsname{\def\PY@tc##1{\textcolor[rgb]{0.00,0.63,0.00}{##1}}}
\expandafter\def\csname PY@tok@gr\endcsname{\def\PY@tc##1{\textcolor[rgb]{1.00,0.00,0.00}{##1}}}
\expandafter\def\csname PY@tok@ge\endcsname{\let\PY@it=\textit}
\expandafter\def\csname PY@tok@gs\endcsname{\let\PY@bf=\textbf}
\expandafter\def\csname PY@tok@gp\endcsname{\let\PY@bf=\textbf\def\PY@tc##1{\textcolor[rgb]{0.00,0.00,0.50}{##1}}}
\expandafter\def\csname PY@tok@go\endcsname{\def\PY@tc##1{\textcolor[rgb]{0.53,0.53,0.53}{##1}}}
\expandafter\def\csname PY@tok@gt\endcsname{\def\PY@tc##1{\textcolor[rgb]{0.00,0.27,0.87}{##1}}}
\expandafter\def\csname PY@tok@err\endcsname{\def\PY@bc##1{\setlength{\fboxsep}{0pt}\fcolorbox[rgb]{1.00,0.00,0.00}{1,1,1}{\strut ##1}}}
\expandafter\def\csname PY@tok@kc\endcsname{\let\PY@bf=\textbf\def\PY@tc##1{\textcolor[rgb]{0.00,0.50,0.00}{##1}}}
\expandafter\def\csname PY@tok@kd\endcsname{\let\PY@bf=\textbf\def\PY@tc##1{\textcolor[rgb]{0.00,0.50,0.00}{##1}}}
\expandafter\def\csname PY@tok@kn\endcsname{\let\PY@bf=\textbf\def\PY@tc##1{\textcolor[rgb]{0.00,0.50,0.00}{##1}}}
\expandafter\def\csname PY@tok@kr\endcsname{\let\PY@bf=\textbf\def\PY@tc##1{\textcolor[rgb]{0.00,0.50,0.00}{##1}}}
\expandafter\def\csname PY@tok@bp\endcsname{\def\PY@tc##1{\textcolor[rgb]{0.00,0.50,0.00}{##1}}}
\expandafter\def\csname PY@tok@fm\endcsname{\def\PY@tc##1{\textcolor[rgb]{0.00,0.00,1.00}{##1}}}
\expandafter\def\csname PY@tok@vc\endcsname{\def\PY@tc##1{\textcolor[rgb]{0.10,0.09,0.49}{##1}}}
\expandafter\def\csname PY@tok@vg\endcsname{\def\PY@tc##1{\textcolor[rgb]{0.10,0.09,0.49}{##1}}}
\expandafter\def\csname PY@tok@vi\endcsname{\def\PY@tc##1{\textcolor[rgb]{0.10,0.09,0.49}{##1}}}
\expandafter\def\csname PY@tok@vm\endcsname{\def\PY@tc##1{\textcolor[rgb]{0.10,0.09,0.49}{##1}}}
\expandafter\def\csname PY@tok@sa\endcsname{\def\PY@tc##1{\textcolor[rgb]{0.73,0.13,0.13}{##1}}}
\expandafter\def\csname PY@tok@sb\endcsname{\def\PY@tc##1{\textcolor[rgb]{0.73,0.13,0.13}{##1}}}
\expandafter\def\csname PY@tok@sc\endcsname{\def\PY@tc##1{\textcolor[rgb]{0.73,0.13,0.13}{##1}}}
\expandafter\def\csname PY@tok@dl\endcsname{\def\PY@tc##1{\textcolor[rgb]{0.73,0.13,0.13}{##1}}}
\expandafter\def\csname PY@tok@s2\endcsname{\def\PY@tc##1{\textcolor[rgb]{0.73,0.13,0.13}{##1}}}
\expandafter\def\csname PY@tok@sh\endcsname{\def\PY@tc##1{\textcolor[rgb]{0.73,0.13,0.13}{##1}}}
\expandafter\def\csname PY@tok@s1\endcsname{\def\PY@tc##1{\textcolor[rgb]{0.73,0.13,0.13}{##1}}}
\expandafter\def\csname PY@tok@mb\endcsname{\def\PY@tc##1{\textcolor[rgb]{0.40,0.40,0.40}{##1}}}
\expandafter\def\csname PY@tok@mf\endcsname{\def\PY@tc##1{\textcolor[rgb]{0.40,0.40,0.40}{##1}}}
\expandafter\def\csname PY@tok@mh\endcsname{\def\PY@tc##1{\textcolor[rgb]{0.40,0.40,0.40}{##1}}}
\expandafter\def\csname PY@tok@mi\endcsname{\def\PY@tc##1{\textcolor[rgb]{0.40,0.40,0.40}{##1}}}
\expandafter\def\csname PY@tok@il\endcsname{\def\PY@tc##1{\textcolor[rgb]{0.40,0.40,0.40}{##1}}}
\expandafter\def\csname PY@tok@mo\endcsname{\def\PY@tc##1{\textcolor[rgb]{0.40,0.40,0.40}{##1}}}
\expandafter\def\csname PY@tok@ch\endcsname{\let\PY@it=\textit\def\PY@tc##1{\textcolor[rgb]{0.25,0.50,0.50}{##1}}}
\expandafter\def\csname PY@tok@cm\endcsname{\let\PY@it=\textit\def\PY@tc##1{\textcolor[rgb]{0.25,0.50,0.50}{##1}}}
\expandafter\def\csname PY@tok@cpf\endcsname{\let\PY@it=\textit\def\PY@tc##1{\textcolor[rgb]{0.25,0.50,0.50}{##1}}}
\expandafter\def\csname PY@tok@c1\endcsname{\let\PY@it=\textit\def\PY@tc##1{\textcolor[rgb]{0.25,0.50,0.50}{##1}}}
\expandafter\def\csname PY@tok@cs\endcsname{\let\PY@it=\textit\def\PY@tc##1{\textcolor[rgb]{0.25,0.50,0.50}{##1}}}

\def\PYZbs{\char`\\}
\def\PYZus{\char`\_}
\def\PYZob{\char`\{}
\def\PYZcb{\char`\}}
\def\PYZca{\char`\^}
\def\PYZam{\char`\&}
\def\PYZlt{\char`\<}
\def\PYZgt{\char`\>}
\def\PYZsh{\char`\#}
\def\PYZpc{\char`\%}
\def\PYZdl{\char`\$}
\def\PYZhy{\char`\-}
\def\PYZsq{\char`\'}
\def\PYZdq{\char`\"}
\def\PYZti{\char`\~}
% for compatibility with earlier versions
\def\PYZat{@}
\def\PYZlb{[}
\def\PYZrb{]}
\makeatother


    % Exact colors from NB
    \definecolor{incolor}{rgb}{0.0, 0.0, 0.5}
    \definecolor{outcolor}{rgb}{0.545, 0.0, 0.0}



    
    % Prevent overflowing lines due to hard-to-break entities
    \sloppy 
    % Setup hyperref package
    \hypersetup{
      breaklinks=true,  % so long urls are correctly broken across lines
      colorlinks=true,
      urlcolor=urlcolor,
      linkcolor=linkcolor,
      citecolor=citecolor,
      }
    % Slightly bigger margins than the latex defaults
    
    \geometry{verbose,tmargin=1in,bmargin=1in,lmargin=1in,rmargin=1in}
    
    

    \begin{document}
    
    
    \maketitle
    
    

    
    \section{Trapezoidal Rule}\label{trapezoidal-rule}

\subsection{Non-composite trapezoidal
rule}\label{non-composite-trapezoidal-rule}

We will be considering a definite integral
\[\displaystyle\int_a^b f(x)\,dx\] with a function \(f(x)\) defined on
the interval \([a,b]\) and having enough smoothness (continuous
derivatives) as needed for various error estimates.

The basic idea is to replace the function \(f(x)\) on the interval
\([a,b]\) by a linear function (whose graph is the secant line
connecting the points \((a,f(a))\) and \((b,f(b))\)) and approximate the
integral of \(f(x)\) by the integral of the linear function.

Below is an example with \(f(x)=e^{-x^2}\) on the interval \([0,2]\).
The integral of \(f(x)\) over \([0,2]\) (which turns out to be about
\(0.88208139\)) will be approximated by the area of the red trapezoid,
which is easy to compute:

\begin{equation*}
  \int_0^2 e^{-x^2}\,dx
  \approx
  \frac{f(0)+f(2)}{2}(2-0)
  =
  1+e^{-4}
  \approx
  1.01831563.
\end{equation*}

The visual representation below confirms that the trapezoidal rule will
produce an overestimate of the actual value.

    \begin{Verbatim}[commandchars=\\\{\}]
{\color{incolor}In [{\color{incolor}2}]:} \PY{n}{f} \PY{p}{=} \PY{n+nb}{inline}\PY{p}{(}\PY{l+s}{\PYZdq{}exp(\PYZhy{}x.\PYZca{}2)\PYZdq{}}\PY{p}{)}\PY{p}{;}
        \PY{n}{x\PYZus{}all} \PY{p}{=} \PY{o}{\PYZhy{}}\PY{l+m+mf}{0.2}\PY{p}{:}\PY{l+m+mf}{0.01}\PY{p}{:}\PY{l+m+mf}{2.2}\PY{p}{;}
        \PY{n}{x\PYZus{}int} \PY{p}{=} \PY{l+m+mi}{0}\PY{p}{:}\PY{l+m+mi}{2}\PY{p}{:}\PY{l+m+mi}{2}\PY{p}{;}
        \PY{n+nb}{fill}\PY{p}{(}\PY{p}{[}\PY{n}{x\PYZus{}int}\PY{p}{,} \PY{n+nb}{fliplr}\PY{p}{(}\PY{n}{x\PYZus{}int}\PY{p}{)}\PY{p}{]}\PY{p}{,} \PY{p}{[}\PY{n}{f}\PY{p}{(}\PY{n}{x\PYZus{}int}\PY{p}{)}\PY{p}{,} \PY{l+m+mi}{0}\PY{p}{,} \PY{l+m+mi}{0}\PY{p}{]}\PY{p}{,} \PY{l+s}{\PYZdq{}r\PYZdq{}}\PY{p}{)}
        \PY{n+nb}{hold} \PY{n}{on}
        \PY{n+nb}{plot}\PY{p}{(}\PY{n}{x\PYZus{}all}\PY{p}{,} \PY{n}{f}\PY{p}{(}\PY{n}{x\PYZus{}all}\PY{p}{)}\PY{p}{,} \PY{l+s}{\PYZdq{}b\PYZdq{}}\PY{p}{,} \PY{l+s}{\PYZdq{}linewidth\PYZdq{}}\PY{p}{,} \PY{l+m+mi}{4}\PY{p}{,} \PY{n}{x\PYZus{}int}\PY{p}{,} \PY{n}{f}\PY{p}{(}\PY{n}{x\PYZus{}int}\PY{p}{)}\PY{p}{,} \PY{l+s}{\PYZdq{}bo\PYZdq{}}\PY{p}{,} \PY{l+s}{\PYZdq{}linewidth\PYZdq{}}\PY{p}{,} \PY{l+m+mi}{4}\PY{p}{)}
        \PY{n+nb}{axis} \PY{n}{equal}\PY{p}{;}
        \PY{n+nb}{axis}\PY{p}{(}\PY{p}{[}\PY{n+nb}{min}\PY{p}{(}\PY{n}{x\PYZus{}all}\PY{p}{)}\PY{p}{,} \PY{n+nb}{max}\PY{p}{(}\PY{n}{x\PYZus{}all}\PY{p}{)}\PY{p}{,} \PY{l+m+mi}{0}\PY{p}{,} \PY{l+m+mf}{1.2}\PY{p}{]}\PY{p}{)}\PY{p}{;}
        \PY{n+nb}{xlabel}\PY{p}{(}\PY{l+s}{\PYZdq{}x\PYZdq{}}\PY{p}{)}\PY{p}{;} \PY{n+nb}{ylabel}\PY{p}{(}\PY{l+s}{\PYZdq{}y\PYZdq{}}\PY{p}{)}\PY{p}{;}
        \PY{n+nb}{title}\PY{p}{(}\PY{l+s}{\PYZdq{}Plot of e\PYZca{}\PYZob{}\PYZhy{}x\PYZca{}2\PYZcb{} on [0,2] and the trapezoidal rule with one trapezoid\PYZdq{}}\PY{p}{)}
\end{Verbatim}


    \begin{center}
    \adjustimage{max size={0.9\linewidth}{0.9\paperheight}}{output_1_0.png}
    \end{center}
    { \hspace*{\fill} \\}
    
    Using Taylor series, one can show that if \(f''\) is continuous, then

\begin{equation*}
  \int_a^b f(x)\,dx
  =
  \frac{f(a)+f(b)}{2}\,(b-a)
  -
  \frac{1}{12}(b-a)^3f''(\xi)
  \qquad
  \text{for some }
  \xi\in(a,b).
\end{equation*}

The first term in the right-hand side is the area of the corresponding
trapezoid, while the second term can be viewed as the error of the
trapezoidal approximation, so we see that the error of the trapezoidal
approximation is given by

\begin{equation*}
  -\frac{1}{12}(b-a)^3f''(\xi)
  \qquad
  \text{for some }
  \xi\in(a,b).
\end{equation*}

    Applying this to our example with \(f(x)=e^{-x^2}\) on the interval
\([0,2]\), we have \(f''(x)=(4x^2-2)e^{-x^2}\) and its absolute value
plotted on the interval \([0,2]\) is given below. We see that on this
interval the maximum value of \(\vert f''\vert\) is \(2\) and thus our
approximation has to have an error that is at most

\begin{equation*}
  \frac{1}{12}(b-a)^3\max_{x\in[a,b]}\vert f''(x)\vert
  =
  \frac{1}{12}(2-0)^3\times2
  =
  \frac{4}{3}.
\end{equation*}

We easily verify that our error \(\vert0.88208139-1.01831563\vert\) is
definitely smaller than \(4/3\).

    \begin{Verbatim}[commandchars=\\\{\}]
{\color{incolor}In [{\color{incolor}3}]:} \PY{n}{ddf} \PY{p}{=} \PY{n+nb}{inline}\PY{p}{(}\PY{l+s}{\PYZdq{}(4*x.\PYZca{}2\PYZhy{}2).*exp(\PYZhy{}x.\PYZca{}2)\PYZdq{}}\PY{p}{)}\PY{p}{;}
        \PY{n}{x} \PY{p}{=} \PY{l+m+mi}{0}\PY{p}{:}\PY{l+m+mf}{0.01}\PY{p}{:}\PY{l+m+mi}{2}\PY{p}{;}
        \PY{n+nb}{plot}\PY{p}{(}\PY{n}{x}\PY{p}{,} \PY{n+nb}{abs}\PY{p}{(}\PY{n}{ddf}\PY{p}{(}\PY{n}{x}\PY{p}{)}\PY{p}{)}\PY{p}{,} \PY{l+s}{\PYZdq{}b\PYZhy{}\PYZdq{}}\PY{p}{,} \PY{l+s}{\PYZdq{}linewidth\PYZdq{}}\PY{p}{,} \PY{l+m+mi}{4}\PY{p}{)}
        \PY{n+nb}{xlabel}\PY{p}{(}\PY{l+s}{\PYZdq{}x\PYZdq{}}\PY{p}{)}\PY{p}{;} \PY{n+nb}{ylabel}\PY{p}{(}\PY{l+s}{\PYZdq{}y\PYZdq{}}\PY{p}{)}\PY{p}{;}
        \PY{n+nb}{title}\PY{p}{(}\PY{l+s}{\PYZdq{}Plot of the absolute value of the second derivative of e\PYZca{}\PYZob{}\PYZhy{}x\PYZca{}2\PYZcb{} on [0,2]\PYZdq{}}\PY{p}{)}
\end{Verbatim}


    \begin{center}
    \adjustimage{max size={0.9\linewidth}{0.9\paperheight}}{output_4_0.png}
    \end{center}
    { \hspace*{\fill} \\}
    
    \subsection{Composite trapezoidal
rule}\label{composite-trapezoidal-rule}

In practice, we do not use the trapezoidal rule with just one trapezoid.
Instead, the interval \([a,b]\) is divided into a partition
\(a=x_0<x_1<x_2<\cdots<x_n=b\) and the integral of \(f(x)\) over the
interval \([a,b]\) is approximated by the sum of the areas of the
trapezoids formed on each subinterval \([x_{i-1},x_i]\):

\begin{equation*}
  \int_a^b f(x)\,dx
  \approx
  \sum_{i=1}^n\frac{f(x_{i-1})+f(x_i)}{2}(x_i-x_{i-1}).
\end{equation*}

Below is an example showing a partition of \([0,2]\) into four
subintervals (of equal length \(0.5\)) and the depiction of the
trapezoidal rule (colored red). The computed approximation is
\(0.88061863\); notice that in this case the error of the approximation
is \[\vert0.88208139-0.88061863\vert\approx1.46\times10^{-3}.\]

    \begin{Verbatim}[commandchars=\\\{\}]
{\color{incolor}In [{\color{incolor}4}]:} \PY{n}{f} \PY{p}{=} \PY{n+nb}{inline}\PY{p}{(}\PY{l+s}{\PYZdq{}exp(\PYZhy{}x.\PYZca{}2)\PYZdq{}}\PY{p}{)}\PY{p}{;}
        \PY{n}{x\PYZus{}all} \PY{p}{=} \PY{o}{\PYZhy{}}\PY{l+m+mf}{0.2}\PY{p}{:}\PY{l+m+mf}{0.01}\PY{p}{:}\PY{l+m+mf}{2.2}\PY{p}{;}
        \PY{n}{x\PYZus{}int} \PY{p}{=} \PY{l+m+mi}{0}\PY{p}{:}\PY{l+m+mf}{0.5}\PY{p}{:}\PY{l+m+mi}{2}\PY{p}{;}
        \PY{n+nb}{fill}\PY{p}{(}\PY{p}{[}\PY{n}{x\PYZus{}int}\PY{p}{,} \PY{n+nb}{fliplr}\PY{p}{(}\PY{n}{x\PYZus{}int}\PY{p}{)}\PY{p}{]}\PY{p}{,} \PY{p}{[}\PY{n}{f}\PY{p}{(}\PY{n}{x\PYZus{}int}\PY{p}{)}\PY{p}{,} \PY{n+nb}{zeros}\PY{p}{(}\PY{l+m+mi}{1}\PY{p}{,}\PY{n+nb}{length}\PY{p}{(}\PY{n}{x\PYZus{}int}\PY{p}{)}\PY{p}{)}\PY{p}{]}\PY{p}{,} \PY{l+s}{\PYZdq{}r\PYZdq{}}\PY{p}{)}
        \PY{n+nb}{hold} \PY{n}{on}
        \PY{n+nb}{plot}\PY{p}{(}\PY{n}{x\PYZus{}all}\PY{p}{,} \PY{n}{f}\PY{p}{(}\PY{n}{x\PYZus{}all}\PY{p}{)}\PY{p}{,} \PY{l+s}{\PYZdq{}b\PYZdq{}}\PY{p}{,} \PY{l+s}{\PYZdq{}linewidth\PYZdq{}}\PY{p}{,} \PY{l+m+mi}{4}\PY{p}{,} \PY{n}{x\PYZus{}int}\PY{p}{,} \PY{n}{f}\PY{p}{(}\PY{n}{x\PYZus{}int}\PY{p}{)}\PY{p}{,} \PY{l+s}{\PYZdq{}bo\PYZdq{}}\PY{p}{,} \PY{l+s}{\PYZdq{}linewidth\PYZdq{}}\PY{p}{,} \PY{l+m+mi}{4}\PY{p}{)}
        \PY{n+nb}{axis} \PY{n}{equal}\PY{p}{;}
        \PY{n+nb}{axis}\PY{p}{(}\PY{p}{[}\PY{n+nb}{min}\PY{p}{(}\PY{n}{x\PYZus{}all}\PY{p}{)}\PY{p}{,} \PY{n+nb}{max}\PY{p}{(}\PY{n}{x\PYZus{}all}\PY{p}{)}\PY{p}{,} \PY{l+m+mi}{0}\PY{p}{,} \PY{l+m+mf}{1.2}\PY{p}{]}\PY{p}{)}\PY{p}{;}
        \PY{n+nb}{xlabel}\PY{p}{(}\PY{l+s}{\PYZdq{}x\PYZdq{}}\PY{p}{)}\PY{p}{;} \PY{n+nb}{ylabel}\PY{p}{(}\PY{l+s}{\PYZdq{}y\PYZdq{}}\PY{p}{)}\PY{p}{;}
        \PY{n+nb}{title}\PY{p}{(}\PY{l+s}{\PYZdq{}Plot of e\PYZca{}\PYZob{}\PYZhy{}x\PYZca{}2\PYZcb{} on [0,2] and the trapezoidal rule with one trapezoid\PYZdq{}}\PY{p}{)}
\end{Verbatim}


    \begin{center}
    \adjustimage{max size={0.9\linewidth}{0.9\paperheight}}{output_6_0.png}
    \end{center}
    { \hspace*{\fill} \\}
    
    \subsection{Composite trapezoidal rule with uniform
spacing}\label{composite-trapezoidal-rule-with-uniform-spacing}

For simplicity of implementation of the trapezoidal rule, uniform
spacing is often used in which \(x_i-x_{i-1}=\dfrac{b-a}{n}\), a
quantity denoted by \(h\). In this case the expression for the
trapezoidal rule simplifies and becomes

\begin{align*}
  \int_a^b f(x)\,dx
  &\approx
  \sum_{i=1}^n\frac{f(x_{i-1})+f(x_i)}{2}(x_i-x_{i-1})\\
  &=
  \frac{h}{2}\sum_{i=1}^n\left(f(x_{i-1})+f(x_i)\right)\\
  &=
  \frac{h}{2}\left(f(x_0)+2f(x_1)+2f(x_2)+\dots+2f(x_{n-1})+f(x_n)\right),
\end{align*}

which is often denoted by \(T_n\) (trapezoidal rule with \(n\) equal
subintervals).

If \(f''\) is continuous, then the error of the approximation can be
shown to satisfy

\begin{equation*}
  \int_a^b f(x)\,dx
  =
  T_n
  -
  \frac{b-a}{12}f''(\xi)h^2
  \quad
  \text{for some }
  \xi\in(a,b),
\end{equation*}

that is, the error is on the order of \(\mathcal{O}(h^2)\).

    \paragraph{Example}\label{example}

Suppose we want to approximate \(\displaystyle\int_0^2 xe^{-x}\,dx\)
using the composite trapezoidal rule with an error less than
\(\dfrac{1}{2}\times10^{-4}\) (i.e., so that the first four decimal
places are correct). What should \(n\) be? What is the computed
approximation? What is the actual error?

\subparagraph{Solution}\label{solution}

With \(a=0\) and \(b=2\), the error will be

\begin{equation*}
  -\frac{b-a}{12}f''(\xi)h^2
  =
  -\frac{b-a}{12}f''(\xi)\left(\frac{b-a}{n}\right)^2
  =
  -\frac{(b-a)^3}{12n^2}f''(\xi).
\end{equation*}

Since \(f''(x)=(x-2)e^{-x}\) (check!), we can check that on \([0,2]\)
the maximum of \(\vert f''\vert\) is \(2\). That means that the error
estimate satisfies

\begin{equation*}
  \left\vert -\frac{(b-a)^3}{12n^2}f''(\xi)\right\vert
  \le
  \frac{8}{12n^2}\times2
  =
  \frac{4}{3n^2}.
\end{equation*}

If we can find \(n\) that satisfies

\begin{equation*}
  \frac{4}{3n^2}
  \le
  \frac{1}{2}\times10^{-4},
\end{equation*}

then we will also satisfy that the error will for sure be less than or
equal to \(\dfrac{1}{2}\times10^{-4}\). The above inequality leads to
\(n\ge163.299\), so we can take \(n=164\).

Below we see a computation using our implemented trapezoidal rule. We
see that \(T_{164}=0.593980079694997\) in format long. The exact answer
for the integral is \(1-\dfrac{3}{e^2}\approx0.593994150290162\).
Checking the error directly we see that it is about \(1.4\times10^{-5}\)
which satisfies our requirement that it be less than
\(\dfrac{1}{2}\times10^{-4}=5\times10^{-5}\).

    \begin{Verbatim}[commandchars=\\\{\}]
{\color{incolor}In [{\color{incolor}5}]:} \PY{c}{\PYZpc{} Define the exact value}
        \PY{n}{integral} \PY{p}{=} \PY{l+m+mi}{1}\PY{o}{\PYZhy{}}\PY{l+m+mi}{3}\PY{o}{/}\PY{n+nb}{e}\PYZca{}\PY{l+m+mi}{2}\PY{p}{;}
        
        \PY{c}{\PYZpc{} Call the trapezoidal rule}
        \PY{n}{trap} \PY{p}{=} \PY{n}{Trapezoid}\PY{p}{(}\PY{p}{@}\PY{p}{(}\PY{n}{x}\PY{p}{)} \PY{n}{x}\PY{o}{.*}\PY{n+nb}{exp}\PY{p}{(}\PY{o}{\PYZhy{}}\PY{n}{x}\PY{p}{)}\PY{p}{,} \PY{l+m+mi}{0}\PY{p}{,} \PY{l+m+mi}{2}\PY{p}{,} \PY{l+m+mi}{164}\PY{p}{)}\PY{p}{;}
        
        \PY{c}{\PYZpc{} Print out results}
        \PY{n+nb}{printf}\PY{p}{(}\PY{l+s}{\PYZdq{}\PYZbs{}nExact value of the integral is    \PYZpc{}17.15f.\PYZbs{}n\PYZdq{}}\PY{p}{,} \PY{n}{integral}\PY{p}{)}
        \PY{n+nb}{printf}\PY{p}{(}\PY{l+s}{\PYZdq{}Computed value of the integral is \PYZpc{}17.15f.\PYZbs{}n\PYZdq{}}\PY{p}{,} \PY{n}{trap}\PY{p}{)}
        \PY{n+nb}{printf}\PY{p}{(}\PY{l+s}{\PYZdq{}Absolute value of the error is    \PYZpc{}.3e.\PYZbs{}n\PYZdq{}}\PY{p}{,} \PY{n}{integral} \PY{o}{\PYZhy{}} \PY{n}{trap}\PY{p}{)}
\end{Verbatim}


    \begin{Verbatim}[commandchars=\\\{\}]
Time elapsed = 0.000026 seconds.

Exact value of the integral is    0.593994150290162.
Computed value of the integral is 0.593980079694997.
Absolute value of the error is    1.407e-05.

    \end{Verbatim}


    % Add a bibliography block to the postdoc
    
    
    
    \end{document}
